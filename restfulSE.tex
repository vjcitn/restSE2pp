%\documentclass[twocolumn,10pt]{article}
%\usepackage{hyperref}

\documentclass[applications]{gen-bioinformatics}

\makeatletter
\renewcommand\boldmath{\@nomath\boldmath\mathversioo
n{bold}}
\makeatother

\usepackage{amssymb,amsfonts,url,times}
\usepackage{graphics}
\usepackage{amsmath}
\usepackage{dcolumn}
\newcolumntype{.}{D{.}{.}{-1}}
%\usepackage{hlight}

\urlstyle{rm}
\def\email#1{#1}



\newcommand{\Rfunction}[1]{{\texttt{#1}}}
\newcommand{\Robject}[1]{{\texttt{#1}}}
\newcommand{\Rpackage}[1]{{\textit{#1}}}
\newcommand{\Rmethod}[1]{{\texttt{#1}}}
\newcommand{\Rfunarg}[1]{{\texttt{#1}}}
\newcommand{\Rclass}[1]{{\textit{#1}}}
\providecommand{\OO}[1]{\operatorname{O}\left(#1\right)}
 

\author[1]{\pfnm{Shweta}
  \pinit{}
  \psnm{Gopaulakrishnan}}

\author[1]{\pfnm{Samuela}
  \pinit{}
  \psnm{Pollack}}

\author[1]{\pfnm{Benjamin}
  \pinit{}
  \psnm{Stubbs}}

\author[1]{\pfnm{Vincent}
  \pinit{J}
  \psnm{Carey}}

\address[1]{\porgdiv{Channing Division of Network Medicine}
  \porgname{Brigham and Women's Hospital}
  \pstreet{181 Longwood Avenue }
  \pcity{Boston}
  \postcode{02115}
  \pcnty{USA}}


 

\begin{document}


\title{restfulSE: a semantically rich interface for cloud-scale genomics}
\maketitle

\begin{abstract}
\begin{subabstract}[Summary]
Bioconductor's \verb+SummarizedExperiment+ class
unites numerical assay quantifications with sample- and
experiment-level metadata.  We describe a deployment
of this data model for data resources that are accessible
via REST APIs.  We illustrate \verb+SummarizedExperiment+s with
HDF Server, HDF Cloud, and Google BigQuery back ends.
\end{subabstract}
\begin{subabstract}[Availability] Package \Rpackage{restfulSE} of Bioconductor
 (\url {www.bioconductor.org}). Open source.
\end{subabstract}
\begin{subabstract}[Contact]reshg@channing.harvard.edu
\end{subabstract}
%\begin{subabstract}[keywords] REST API, HDF5, Bioconductor
\end{abstract}
\section*{Introduction}

The analysis of multiomic archives (like TCGA) and single-cell
transcriptomic experiments (like the 10x 1.3 million mouse neuron
dataset) typically begin with downloads of large files and
conversion of file contents into formats based on local preferences.
In this paper we consider how targeted queries of large remote
genomic data resources can be conducted using methods available
for Bioconductor's \verb+SummarizedExperiment+ class.
This approach can be used to centralize
large data archives and diminish redundant disk
consumption.  Clients for HDF5 data are available in
numerous language; our Bioconductor interface permits access to
remote HDF5 archives with
familiar and semantically meaningful programmatic idioms.

\section*{Description}

\subsection*{The SummarizedExperiment class and related methods}

Let $X$ denote a matrix of quantifications arising from a genome
scale assay with $G$ assay features measured on $N$ experimental
samples.  In the 10x mouse neuron dataset, $G = 27998$ and $N= 1.3$ million.
When these quantifications are managed in a Bioconductor \verb+SummarizedExperiment X+, the numbers in $X$ are programmatically bound to a $G \times F$
table of feature-level metadata (e.g., gene or transcript names and
characteristics) accessible by the \verb+rowData+ method, and to an $N \times R$ table of sample-level metadata accessible by \verb+colData+. 
[CITE HUBER] The idiom \verb+X[G,S]+ expresses filtering of 
the information
to features \verb+G+ and samples \verb+S+.  A \verb+GRanges+ 
instance [CITE LAWRENCE] defining genomic coordinates for features may be bound to \verb+X+,
allowing the use of \verb+subsetByOverlaps+ to isolate features
coincident with or near the elements of a given \verb+GRanges+.
This idea is often used when seeking patterns of variation in
assay elements near regions identified as peaks of factor binding.

\subsection*{Managing quantifications remotely}

\textbf{HDF Server.}  The \verb+h5serv+ Python package defines a Tornado-based RESTful web
service for providing access to HDF5 datasets.  HDF5 representations of
numerical matrices encoding genome scale assays are 
frequently encountered.  Because
the server design permits compression, and to facilitate direct natural 
slicing of matrix data, our examples use a dense representation.  The
HDF Server API handles access to data and metadata.  Parallel query
resolution is supported on suitable hardware.

\textbf{HDF Cloud.}  An S3-based distributed data model for HDF5
datasets has been implemented by the HDF Group.

\textbf{Google BigQuery.}  

\subsection*{Semantic value through hybrid design.}

The \verb+restfulSE+ package provides interfaces to remote HDF5
and BigQuery so that the numerical content lodged in these services
satisfies the Bioconductor \verb+DelayedArray+ API.  
Any \verb+DelayedArray+ instance can serve as the \verb+assay+
component of a \verb+SummarizedExperiment+ instance.  Thus the
capacities of \verb+SummarizedExperiment+ to bind semantically
rich metadata to genome-scale assays are extended implicitly to
data resources that are intrinsically metadata-poor.  In 
conjunction with the \verb+rhdf5client+ and \verb+bigrquery+ packages,
\verb+restfulSE+ translates filtering and selection operations
which are readily defined using \verb+rowData+, \verb+rowRanges+,
and \verb+colData+ into formal queries resolvable by the HDF5 and
BigQuery services.  Numerical results are transmitted as needed.

\section*{Performance}

\section*{Future enhancements}

\section*{Acknowledgments}
Support for the development of this software was provided by grant
(NIH) P20 CA096470 - Pi Wing Wong Center for Computational Study of
Biological Systems.
\bibliography{BioC}

\end{document}
